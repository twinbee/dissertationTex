%\setcounter{glosscount}{\thepage}
%\setcounter{glosscount}{13}
\setcounter{chapter}{0}

\chapter*{NOTATION AND GLOSSARY}
\section*{General Usage and Terminology}
\thispagestyle{plain}
The notation used in this text represents fairly standard mathematical
and computational usage. In many cases these fields tend to use
different preferred notation to indicate the same concept, and these
have been reconciled to the extent possible,
given the interdisciplinary nature of the material.
In particular, the notation for partial
derivatives varies extensively, and the notation used is chosen
for stylistic convenience based on the application. While it would be
convenient to utilize a standard nomenclature for this important
symbol, the many alternatives currently in the published literature
will continue to be utilized.



The blackboard fonts are used to denote standard
sets of numbers: $\real$ for the field of real numbers,
$\complex$ for the complex field, $\integer$ for the integers,
and $\rational$ for the rationals.
The capital letters, $A, B, \cdots $ are used to denote matrices,
including capital greek letters, e.g., $\Lambda$ for a diagnonal matrix.
Functions which are denoted in boldface type typically represent 
vector valued functions, and real valued functions usually are set in
lower case roman or greek letters.
Caligraphic letters, e.g., $\cal V$, are used to denote spaces such
as $\cal V$ denoting a vector space, $\cal H$ denoting a Hilbert space,
or $\cal F$ denoting a general function space.
Lower case letters such as $i, j, k, l, m, n$ and sometimes $p$  and $d$
are used to denote indices.

Vectors are typset in square brackets, e.g., $[ \cdot ]$, and matrices are
typeset in parenthesese, e.g., $( \cdot)$. 
In general the norms are typeset using double pairs of
lines, e.g., $|| \cdot ||$,
and the abolute value of numbers is denoted using a single pairs of lines,
e.g., $| \cdot |$.
Single pairs of lines around matrices indicates the determinant of the
matrix.

%\newpage

\glossfont



