%~~~~~~~~~~~~~~~~~~~~~~~~~~~~~~~~~~~~~~~~~~~~~~~~~~~~~~~~~~~~~~~~~~~~~~~~~~~~~~~
%DOCUMENT:      DISSc.tex
%AUTHOR:    J. Kolibal
%REV:           26
%DATE:          Mon Nov 21 09:40:52 CST 2005


%
%
%DEPENDENCIES: Style files and bibliography style files:
%               dissertation_usm.cls  DISS_pream.tex
%               plain.bst
%
%              Chapters in the dissertation:
%               DISS_chap1.tex DISS_chap2.tex DISS_chap3.tex
%                DISS_chap4.tex DISS_chap5.tex DISS_chap6.tex  DISS_output.tex
%
%              The bibliography database:
%               DISSc.bib
%
%              Disseration required lead pages:
%               DISS_abstract.tex DISS_abstract_titlepage.tex
%               DISS_titlepage.tex DISS_abbreviations.tex
%
%              Figures:
%               DISS_mach.PS DISS_conv.PS DISS_machgnu.PS
%
%              Verbatim computer input:
%               DISS_output.tex
%
%PURPOSE:      To demonstrate the generation of a dissertation for
%           for Scientific Computing and Mathematics at USM.
%
%APPROACH:     This uses modified class files based on book.cls files
%               to control the layout of the page for the majority of
%               the page appearance. Some commands are not hard coded
%               as macros in the style file. These must be entered
%               as shown in this example in order to produce a file which
%               is consistent with the requirments of the Graduate College.
%
%               You must also be aware of the rules that LaTeX uses to
%               set a page. This includes 1) each paragraph begins on a
%               newline after a blank line; 2) a word begins when at least
%               one blank space is left on a line (punctuation belongs
%               to the end of words, e.g.,  'this sentence ends. ' is
%               the way to type, not 'this sentence ends .'; 3) parentheses
%               belong to the words internal to the bracketed phrase,
%               e.g., 'this (while they said otherwise) and not those'
%               in contrast to 'this( while they said otherwise )and not those';
%               and, 4) remember to tie together objects with forced
%               spaces, thus 'Fig.~5' and not 'Fig. 5'. The detail
%               is important to getting TeX to interpret the spacing.
%
%
%+++++++++++++++++++++THIS VERSION IS SETUP TO RUN ON LINUX+++++++++++++++++++++

%
% You need to run teTeX (LaTeX2e) with the paths set correctly as
%TEXINPUTS=.:$HOME/common/defaults/latex/inputs//:/home/defaults/latex/inputs//:
%
% If you are at USM on any Scientific Computing workstation running  Linux,
% the setup of redhat linux should work without any additions modifations.
%
%++++++++++++++++++++++SETUP THE MARGINS AND SPACING++++++++++++++++++++++++++++

% THIS SETUP INCLUDES THE AMSLatex FONTS AND MACROS AND THE NEW graphicx PACKAGE
\documentclass[oneside,12pt]{dissertation_usm}
\usepackage{graphicx,amssymb,amsfonts,amsmath,amsthm,eucal}
% THE eucal package provides improved math caligraphic fonts.

% PROVIDES AN ALTERNATIVE FONT ENCODING.
%\usepackage[T1]{fontenc}

%THIS USES THE Times-Roman FONTS (DEFAULT IS TO USE THIS PACKAGE)
\usepackage{mathptm}

% SELECT GLOSSARY ---
%THIS PACKAGE ALLOWS YOU TO CREATE GLOSSARY ENTRIES. DO NOT UNCOMMENT EVEN
% IF YOU PLAN TO NOT USE.
\usepackage{nomencl}
% SELECT GLOSSARY ---


%THIS ALLOWS YOU TO CHANGE THE SCALE OF YOUR OUTPUT (THE UPPER LEFT CORNER
% IS HELD CONSTANT). DO NOT USE UNLESS YOU NEED TO SCALE YOUR OUTPUT.
% CHANGE THE \mag=VALUE command in the mag.sty FILE TO CHANGE THE SCALE.
%\mag=1000 IS NORMAL SCALING.
%\usepackage{mag}
%



%PROVIDES COLOR FOR TEXT AND BACKGROUND.
\usepackage{color}


%% THIS SETUP IS FOR Latex WITHOUT ANY AMSLatex.
%\documentclass[oneside,12pt]{dissertation_usm}

%\usepackage{graphicx}


%
\setlength{\topmargin}{-0.1in}
%\setlength{\textheight}{8.9in}
\setlength{\textheight}{9.3in}
\setlength{\textwidth}{5.9in}
\setlength{\oddsidemargin}{0.5in}
\setlength{\evensidemargin}{0.5in}
%






%YOU MUST BRING IN THE epsf.tex FILE IF YOU USE epsfbox TO INPUT YOUR FIGURES.
\input{epsf}


%\sloppy
%\raggedbottom



%DEFINE COMMANDS. THESE ARE GLOBAL. USE THE DISS_pream_ams IF YOU ARE USING
% AMS LaTeX COMMANDS.
\input{preamble/DISS_pream}
\input{preamble/DISS_pream_ams}



%THIS WILL ALLOW YOU TO MAKE AN INDEX. TAKE THIS COMMAND OUT IF YOU DO
% NOT DESIRE AN INDEX.
\makeindex


%SELECT GLOSSARY ---
%THIS COMMAND WILL ALLOW YOU TO MAKE A GLOSSARY.
% DO NOT USE UNLESS YOU HAVE READ THE DOCUMENTATION FOR THE nomenclature.sty
% PACKAGE.
\refpage
\makeglossary
%SELECT GLOSSARY ---







\begin{document}




%THE TEXT OF THE DOCUMENT.





%DEFINE THE LEADING PAGES.
%THIS IS THE ABSTRACT TITLE PAGE.
\thispagestyle{empty}
\input{preamble/DISS_abstract_titlepage}
\setlength{\textheight}{8.7in}
\newpage


%THIS IS THE ABSTRACT.
\setcounter{page}{1}   %THIS PAGE MUST HAVE AN ARABIC NUMBER 1 ON IT
\thispagestyle{plain}
\input{preamble/DISS_abstract}
\newpage


%THIS IS THE COPYRIGHT PAGE.
\setcounter{page}{1}
\thispagestyle{empty}
\input{preamble/DISS_copyright}
\newpage




%THIS IS THE TITLE PAGE.
\thispagestyle{empty}
\input{preamble/DISS_titlepage}
\newpage


% THE ROMAN NUMBER OF THE PAGES BEGINS HERE
% THIS IS THE SECOND PAGE (ii) OF THE DOCUMENT.
{\pagestyle{plain}
\pagenumbering{roman}
\setcounter{page}{2}

%DEFINE THE LEADING PAGES.
%THIS IS THE DEDICATION PAGE.
% THIS IS THE SECOND PAGE (ii) OF THE DOCUMENT.
\input{preamble/DISS_dedication}
\newpage

%THIS IS THE TABLE OF CONTENTS.
% THIS IS THE SECOND PAGE (ii) OF THE DOCUMENT IF THERE
% IS NO DEDICATION PAGE.
%\thispagestyle{plain}
%\pagenumbering{roman}
%\setcounter{page}{2}
\tableofcontents

%ADDED ABSTRACT PAGE------------------------------------------------------------
\newcounter{abspage}
\setcounter{abspage}{1}
\addtocontents{toc}{ \addvspace{15pt}}

%PULLED TOC ENTRIES FLUSH.
\addtocontents{toc}{ {\bf ABSTRACT}
      \dtfil \hspace*{10pt}  \arabic{abspage}}
\addtocontents{toc}
%ADDED ABSTRACT PAGE------------------------------------------------------------


%THIS IS THE LISTING OF THE ACKNOWLEDGEMENTS
\newcounter{ackpage}
\setcounter{ackpage}{2}
\addtocontents{toc}{ \addvspace{15pt}}

%PULLED TOC ENTRIES FLUSH.
\addtocontents{toc}{ {\bf \hspace*{-21pt} ACKNOWLEDGEMENTS}
      \dtfil   \roman{ackpage}}
\addtocontents{toc} { \addvspace{15pt}}


% * * * MODIFIED listoffigures AND listoftables * * * -------------------------
%PULLED TOC ENTRIES FLUSH.
%Patch if page number of illustration entry is incorrect
\addtocounter{page}{1}
\addtocontents{toc}{\hspace*{-21pt} {\bf LIST OF ILLUSTRATIONS}
      \dtfil \hspace*{10pt}  \roman{page}}
\addtocontents{toc}{ \addvspace{15pt}}
\addtocounter{page}{-1}
\listoffigures


% PULLED TOC ENTRIES FLUSH.
%Patch if page number of tables entry is incorrect
\addtocounter{page}{1}
\addtocontents{toc}{ \addvspace{15pt}}
\addtocontents{toc}{\hspace*{-17pt}{\bf LIST OF TABLES}
       \dtfil \hspace{10pt} \roman{page}}
\addtocounter{page}{-1}
\listoftables
% * * * MODIFIED listoffigures AND listoftables * * * -------------------------


%THIS IS THE LIST OF ABBREVIATIONS.
\clearpage
\thispagestyle{plain}

%PULLED TOC ENTRIES FLUSH.
\addtocontents{toc}{ \addvspace{15pt}}
\addtocontents{toc}{\hspace*{-17pt}{\bf LIST OF ABBREVIATIONS}
           \dtfil \hspace{10pt} \roman{page}}
\addtocontents{toc}{ \addvspace{10pt}}

\input{preamble/DISS_abbreviations}


%SELECT GLOSSARY ---
%THIS IS THE NOMENCLATURE GLOSSARY
\newcounter{glosscount}
%\pagestyle{myheadings}
%\markboth{\small \sl \hfill NOTATION \hfill }{\small \sl \hfill NOTATION  \hfill}
%\setcounter{glosscount}{\thepage}
%\setcounter{glosscount}{13}
\setcounter{chapter}{0}

\chapter*{NOTATION AND GLOSSARY}
\section*{General Usage and Terminology}
\thispagestyle{plain}
The notation used in this text represents fairly standard mathematical
and computational usage. In many cases these fields tend to use
different preferred notation to indicate the same concept, and these
have been reconciled to the extent possible,
given the interdisciplinary nature of the material.
In particular, the notation for partial
derivatives varies extensively, and the notation used is chosen
for stylistic convenience based on the application. While it would be
convenient to utilize a standard nomenclature for this important
symbol, the many alternatives currently in the published literature
will continue to be utilized.



The blackboard fonts are used to denote standard
sets of numbers: $\real$ for the field of real numbers,
$\complex$ for the complex field, $\integer$ for the integers,
and $\rational$ for the rationals.
The capital letters, $A, B, \cdots $ are used to denote matrices,
including capital greek letters, e.g., $\Lambda$ for a diagnonal matrix.
Functions which are denoted in boldface type typically represent 
vector valued functions, and real valued functions usually are set in
lower case roman or greek letters.
Caligraphic letters, e.g., $\cal V$, are used to denote spaces such
as $\cal V$ denoting a vector space, $\cal H$ denoting a Hilbert space,
or $\cal F$ denoting a general function space.
Lower case letters such as $i, j, k, l, m, n$ and sometimes $p$  and $d$
are used to denote indices.

Vectors are typset in square brackets, e.g., $[ \cdot ]$, and matrices are
typeset in parenthesese, e.g., $( \cdot)$. 
In general the norms are typeset using double pairs of
lines, e.g., $|| \cdot ||$,
and the abolute value of numbers is denoted using a single pairs of lines,
e.g., $| \cdot |$.
Single pairs of lines around matrices indicates the determinant of the
matrix.

%\newpage

\glossfont




% THE \label{glosspage} COMMAND MAY HAVE TO BE SET AFTER THE GLOSSARY IS CALLED.
\label{glosspage}
\printglossary[1.2cm]             % SET TO 1.2cm wide.
%\setcounter{glosscount}{13}
%SET PAGE MANUALLY
%ADD GLOSSARY TO TABLE OF CONTENTS
\addtocontents{toc}{ \addvspace{0pt}}
\addtocontents{toc}{\hspace*{-17pt}{\bf NOTATION AND GLOSSARY}
%      \dtfil \hspace*{10pt}  \roman{glosscount}}
      \dtfil \hspace*{10pt}  \pageref{glosspage}}
\addtocontents{toc}{ \addvspace{15pt}}
%SELECT GLOSSARY ---
%\label{glosspage}
%\clearpage
\thispagestyle{empty}



}                                           % CLOSE PLAIN PAGE STYLE.


%THIS ENDS THE PREAMBLE FILES.



%....................................................................................
%THIS CLEARPAGE IS NECESSARY TO CLEAR THE BUFFERS
\clearpage
\baselineskip=18pt


%BEGIN THE TEXT OF THE DOCUMENT.
\pagenumbering{arabic}
\setcounter{page}{1}


%BEGIN THE FIRST CHAPTER WHICH IS IN FILE DISS_chap1.tex IN THE FILE SPACE.
\input{chapter/DISS_chap1}

%BEGIN THE SECOND CHAPTER WHICH IS IN FILE DISS_chap2.tex IN THE FILE SPACE.
\input{chapter/DISS_chap2}


%BEGIN THE THIRD CHAPTER WHICH IS IN FILE DISS_chap3.tex IN THE FILE SPACE.
\input{chapter/DISS_chap3}

%BEGIN THE FOURTH CHAPTER. THIS SHOWS THE USE OF FONTS
%\input{chapter/DISS_chap4}

%BEGIN THE FIFTH CHAPTER. THIS IS TAKE FROM testmath.tex FROM THE AMS PACKAGE.
% TO ILLUSTRATE THE USE OF amslatex.

%\input{chapter/DISS_chap5}


%JK REV 7 MODIFIED ENTRY FOR TABLE OF CONENTS FOR APPENDIX.
\addtocontents{toc}{\addvspace{20pt}}
\addtocontents{toc}{\hspace*{-17pt}{\bf APPENDIX} \hfill }

\input{chapter/DISS_appenda}


%+++++++++++++++++++++++++PROCESS THE BIBLIOGRAPHY++++++++++++++++++++++++++++++





%PROCESS THE BIBLIOGRAPHY
% REVIEW THE DOCUMENTATION IN ./doc.
\baselineskip=12pt


%THE NOCITE COMMAND IS NEEDED TO INPUT REFERENCES WHICH ARE NOT EXPLICITELY
%REFERENCED IN THE BODY OF THE TEXT USING THE CITE COMMAND.



%REV 7
% don't use any of these... ... .. Kolibal refs


%THE NOCITE COMMAND IS NEEDED TO INPUT REFERENCES WHICH ARE NOT EXPLICITELY
%REFERENCED IN THE BODY OF THE TEXT USING THE CITE COMMAND.
%\nocite{mayt}
%\nocite{rozdestvenskiiandjanenko}
%\nocite{Bergermj}
%\nocite{gustafssonb}
%\nocite{abbettmj}
%\nocite{kreissho}
%\nocite{wickelgrenwa}
%\nocite{sodga}
%\nocite{laxpd}
%\nocite{wagons}
%\nocite{bowersandwilson}
%\nocite{krutzm}
%\nocite{lamarshjr}
%\nocite{walkerandmiller}
%\nocite{profioae}
%\nocite{grantpj}
%\nocite{mccauslandi}


{\small
%\baselineskip=12.8pt




%THIS IS OPTIONAL. IT WILL PUT A LISTING OF THE BIBLIOGRAPHY IN THE
% TABLE OF CONTENTS.
\clearpage
\addtocontents{toc}{ \addvspace{15pt}}
\addtocontents{toc}{\hspace*{-17pt}{\bf BIBLIOGRAPHY}
           \dtfil \hspace{10pt} \arabic{page}}
\addtocontents{toc}{ \addvspace{10pt}}


%YOU MAY USE plain OR siam STYLE
\bibliography{DISSc}
%\bibliographystyle{siam}
\bibliographystyle{plain}




%THIS IS OPTIONAL. IT WILL PUT A LISTING OF THE INDEX IN THE
% TABLE OF CONTENTS.
\clearpage
\addtocontents{toc}{ \addvspace{15pt}}
\addtocontents{toc}{\hspace*{-17pt}{\bf INDEX}
           \dtfil \hspace{10pt} \arabic{page}}
\addtocontents{toc}{ \addvspace{10pt}}







}


%++++++++++++++++++++++END PROCESS THE BIBLIOGRAPHY+++++++++++++++++++++++++++++



%++++++++++++++++++++++++++++PROCESS THE INDEX++++++++++++++++++++++++++++++++++
%PROCESS THE INDEX.
% REVIEW THE DOCUMENTATION IN ./doc.

%THIS CHOOSES A WIDE SINGLE SPACE FOR THE INDEX.
\baselineskip=15pt

%~~~~~~~~~~~~~~~~~~~~~~~~~~~~~~~~~~~~~~~~~~~~~~~~~~~~~~~~~~~~~~~~~~~~~~~~~~~~~~~
%DOCUMENT:      DISSc.tex
%AUTHOR:    J. Kolibal
%REV:           26
%DATE:          Mon Nov 21 09:40:52 CST 2005


%
%
%DEPENDENCIES: Style files and bibliography style files:
%               dissertation_usm.cls  DISS_pream.tex
%               plain.bst
%
%              Chapters in the dissertation:
%               DISS_chap1.tex DISS_chap2.tex DISS_chap3.tex
%                DISS_chap4.tex DISS_chap5.tex DISS_chap6.tex  DISS_output.tex
%
%              The bibliography database:
%               DISSc.bib
%
%              Disseration required lead pages:
%               DISS_abstract.tex DISS_abstract_titlepage.tex
%               DISS_titlepage.tex DISS_abbreviations.tex
%
%              Figures:
%               DISS_mach.PS DISS_conv.PS DISS_machgnu.PS
%
%              Verbatim computer input:
%               DISS_output.tex
%
%PURPOSE:      To demonstrate the generation of a dissertation for
%           for Scientific Computing and Mathematics at USM.
%
%APPROACH:     This uses modified class files based on book.cls files
%               to control the layout of the page for the majority of
%               the page appearance. Some commands are not hard coded
%               as macros in the style file. These must be entered
%               as shown in this example in order to produce a file which
%               is consistent with the requirments of the Graduate College.
%
%               You must also be aware of the rules that LaTeX uses to
%               set a page. This includes 1) each paragraph begins on a
%               newline after a blank line; 2) a word begins when at least
%               one blank space is left on a line (punctuation belongs
%               to the end of words, e.g.,  'this sentence ends. ' is
%               the way to type, not 'this sentence ends .'; 3) parentheses
%               belong to the words internal to the bracketed phrase,
%               e.g., 'this (while they said otherwise) and not those'
%               in contrast to 'this( while they said otherwise )and not those';
%               and, 4) remember to tie together objects with forced
%               spaces, thus 'Fig.~5' and not 'Fig. 5'. The detail
%               is important to getting TeX to interpret the spacing.
%
%
%+++++++++++++++++++++THIS VERSION IS SETUP TO RUN ON LINUX+++++++++++++++++++++

%
% You need to run teTeX (LaTeX2e) with the paths set correctly as
%TEXINPUTS=.:$HOME/common/defaults/latex/inputs//:/home/defaults/latex/inputs//:
%
% If you are at USM on any Scientific Computing workstation running  Linux,
% the setup of redhat linux should work without any additions modifations.
%
%++++++++++++++++++++++SETUP THE MARGINS AND SPACING++++++++++++++++++++++++++++

% THIS SETUP INCLUDES THE AMSLatex FONTS AND MACROS AND THE NEW graphicx PACKAGE
\documentclass[oneside,12pt]{dissertation_usm}
\usepackage{graphicx,amssymb,amsfonts,amsmath,amsthm,eucal}
% THE eucal package provides improved math caligraphic fonts.

% PROVIDES AN ALTERNATIVE FONT ENCODING.
%\usepackage[T1]{fontenc}

%THIS USES THE Times-Roman FONTS (DEFAULT IS TO USE THIS PACKAGE)
\usepackage{mathptm}

% SELECT GLOSSARY ---
%THIS PACKAGE ALLOWS YOU TO CREATE GLOSSARY ENTRIES. DO NOT UNCOMMENT EVEN
% IF YOU PLAN TO NOT USE.
\usepackage{nomencl}
% SELECT GLOSSARY ---


%THIS ALLOWS YOU TO CHANGE THE SCALE OF YOUR OUTPUT (THE UPPER LEFT CORNER
% IS HELD CONSTANT). DO NOT USE UNLESS YOU NEED TO SCALE YOUR OUTPUT.
% CHANGE THE \mag=VALUE command in the mag.sty FILE TO CHANGE THE SCALE.
%\mag=1000 IS NORMAL SCALING.
%\usepackage{mag}
%



%PROVIDES COLOR FOR TEXT AND BACKGROUND.
\usepackage{color}


%% THIS SETUP IS FOR Latex WITHOUT ANY AMSLatex.
%\documentclass[oneside,12pt]{dissertation_usm}

%\usepackage{graphicx}


%
\setlength{\topmargin}{-0.1in}
%\setlength{\textheight}{8.9in}
\setlength{\textheight}{9.3in}
\setlength{\textwidth}{5.9in}
\setlength{\oddsidemargin}{0.5in}
\setlength{\evensidemargin}{0.5in}
%






%YOU MUST BRING IN THE epsf.tex FILE IF YOU USE epsfbox TO INPUT YOUR FIGURES.
\input{epsf}


%\sloppy
%\raggedbottom



%DEFINE COMMANDS. THESE ARE GLOBAL. USE THE DISS_pream_ams IF YOU ARE USING
% AMS LaTeX COMMANDS.
\input{preamble/DISS_pream}
\input{preamble/DISS_pream_ams}



%THIS WILL ALLOW YOU TO MAKE AN INDEX. TAKE THIS COMMAND OUT IF YOU DO
% NOT DESIRE AN INDEX.
\makeindex


%SELECT GLOSSARY ---
%THIS COMMAND WILL ALLOW YOU TO MAKE A GLOSSARY.
% DO NOT USE UNLESS YOU HAVE READ THE DOCUMENTATION FOR THE nomenclature.sty
% PACKAGE.
\refpage
\makeglossary
%SELECT GLOSSARY ---







\begin{document}




%THE TEXT OF THE DOCUMENT.





%DEFINE THE LEADING PAGES.
%THIS IS THE ABSTRACT TITLE PAGE.
\thispagestyle{empty}
\input{preamble/DISS_abstract_titlepage}
\setlength{\textheight}{8.7in}
\newpage


%THIS IS THE ABSTRACT.
\setcounter{page}{1}   %THIS PAGE MUST HAVE AN ARABIC NUMBER 1 ON IT
\thispagestyle{plain}
\input{preamble/DISS_abstract}
\newpage


%THIS IS THE COPYRIGHT PAGE.
\setcounter{page}{1}
\thispagestyle{empty}
\input{preamble/DISS_copyright}
\newpage




%THIS IS THE TITLE PAGE.
\thispagestyle{empty}
\input{preamble/DISS_titlepage}
\newpage


% THE ROMAN NUMBER OF THE PAGES BEGINS HERE
% THIS IS THE SECOND PAGE (ii) OF THE DOCUMENT.
{\pagestyle{plain}
\pagenumbering{roman}
\setcounter{page}{2}

%DEFINE THE LEADING PAGES.
%THIS IS THE DEDICATION PAGE.
% THIS IS THE SECOND PAGE (ii) OF THE DOCUMENT.
\input{preamble/DISS_dedication}
\newpage

%THIS IS THE TABLE OF CONTENTS.
% THIS IS THE SECOND PAGE (ii) OF THE DOCUMENT IF THERE
% IS NO DEDICATION PAGE.
%\thispagestyle{plain}
%\pagenumbering{roman}
%\setcounter{page}{2}
\tableofcontents

%ADDED ABSTRACT PAGE------------------------------------------------------------
\newcounter{abspage}
\setcounter{abspage}{1}
\addtocontents{toc}{ \addvspace{15pt}}

%PULLED TOC ENTRIES FLUSH.
\addtocontents{toc}{ {\bf ABSTRACT}
      \dtfil \hspace*{10pt}  \arabic{abspage}}
\addtocontents{toc}
%ADDED ABSTRACT PAGE------------------------------------------------------------


%THIS IS THE LISTING OF THE ACKNOWLEDGEMENTS
\newcounter{ackpage}
\setcounter{ackpage}{2}
\addtocontents{toc}{ \addvspace{15pt}}

%PULLED TOC ENTRIES FLUSH.
\addtocontents{toc}{ {\bf \hspace*{-21pt} ACKNOWLEDGEMENTS}
      \dtfil   \roman{ackpage}}
\addtocontents{toc} { \addvspace{15pt}}


% * * * MODIFIED listoffigures AND listoftables * * * -------------------------
%PULLED TOC ENTRIES FLUSH.
%Patch if page number of illustration entry is incorrect
\addtocounter{page}{1}
\addtocontents{toc}{\hspace*{-21pt} {\bf LIST OF ILLUSTRATIONS}
      \dtfil \hspace*{10pt}  \roman{page}}
\addtocontents{toc}{ \addvspace{15pt}}
\addtocounter{page}{-1}
\listoffigures


% PULLED TOC ENTRIES FLUSH.
%Patch if page number of tables entry is incorrect
\addtocounter{page}{1}
\addtocontents{toc}{ \addvspace{15pt}}
\addtocontents{toc}{\hspace*{-17pt}{\bf LIST OF TABLES}
       \dtfil \hspace{10pt} \roman{page}}
\addtocounter{page}{-1}
\listoftables
% * * * MODIFIED listoffigures AND listoftables * * * -------------------------


%THIS IS THE LIST OF ABBREVIATIONS.
\clearpage
\thispagestyle{plain}

%PULLED TOC ENTRIES FLUSH.
\addtocontents{toc}{ \addvspace{15pt}}
\addtocontents{toc}{\hspace*{-17pt}{\bf LIST OF ABBREVIATIONS}
           \dtfil \hspace{10pt} \roman{page}}
\addtocontents{toc}{ \addvspace{10pt}}

\input{preamble/DISS_abbreviations}


%SELECT GLOSSARY ---
%THIS IS THE NOMENCLATURE GLOSSARY
\newcounter{glosscount}
%\pagestyle{myheadings}
%\markboth{\small \sl \hfill NOTATION \hfill }{\small \sl \hfill NOTATION  \hfill}
%\setcounter{glosscount}{\thepage}
%\setcounter{glosscount}{13}
\setcounter{chapter}{0}

\chapter*{NOTATION AND GLOSSARY}
\section*{General Usage and Terminology}
\thispagestyle{plain}
The notation used in this text represents fairly standard mathematical
and computational usage. In many cases these fields tend to use
different preferred notation to indicate the same concept, and these
have been reconciled to the extent possible,
given the interdisciplinary nature of the material.
In particular, the notation for partial
derivatives varies extensively, and the notation used is chosen
for stylistic convenience based on the application. While it would be
convenient to utilize a standard nomenclature for this important
symbol, the many alternatives currently in the published literature
will continue to be utilized.



The blackboard fonts are used to denote standard
sets of numbers: $\real$ for the field of real numbers,
$\complex$ for the complex field, $\integer$ for the integers,
and $\rational$ for the rationals.
The capital letters, $A, B, \cdots $ are used to denote matrices,
including capital greek letters, e.g., $\Lambda$ for a diagnonal matrix.
Functions which are denoted in boldface type typically represent 
vector valued functions, and real valued functions usually are set in
lower case roman or greek letters.
Caligraphic letters, e.g., $\cal V$, are used to denote spaces such
as $\cal V$ denoting a vector space, $\cal H$ denoting a Hilbert space,
or $\cal F$ denoting a general function space.
Lower case letters such as $i, j, k, l, m, n$ and sometimes $p$  and $d$
are used to denote indices.

Vectors are typset in square brackets, e.g., $[ \cdot ]$, and matrices are
typeset in parenthesese, e.g., $( \cdot)$. 
In general the norms are typeset using double pairs of
lines, e.g., $|| \cdot ||$,
and the abolute value of numbers is denoted using a single pairs of lines,
e.g., $| \cdot |$.
Single pairs of lines around matrices indicates the determinant of the
matrix.

%\newpage

\glossfont




% THE \label{glosspage} COMMAND MAY HAVE TO BE SET AFTER THE GLOSSARY IS CALLED.
\label{glosspage}
\printglossary[1.2cm]             % SET TO 1.2cm wide.
%\setcounter{glosscount}{13}
%SET PAGE MANUALLY
%ADD GLOSSARY TO TABLE OF CONTENTS
\addtocontents{toc}{ \addvspace{0pt}}
\addtocontents{toc}{\hspace*{-17pt}{\bf NOTATION AND GLOSSARY}
%      \dtfil \hspace*{10pt}  \roman{glosscount}}
      \dtfil \hspace*{10pt}  \pageref{glosspage}}
\addtocontents{toc}{ \addvspace{15pt}}
%SELECT GLOSSARY ---
%\label{glosspage}
%\clearpage
\thispagestyle{empty}



}                                           % CLOSE PLAIN PAGE STYLE.


%THIS ENDS THE PREAMBLE FILES.



%....................................................................................
%THIS CLEARPAGE IS NECESSARY TO CLEAR THE BUFFERS
\clearpage
\baselineskip=18pt


%BEGIN THE TEXT OF THE DOCUMENT.
\pagenumbering{arabic}
\setcounter{page}{1}


%BEGIN THE FIRST CHAPTER WHICH IS IN FILE DISS_chap1.tex IN THE FILE SPACE.
\input{chapter/DISS_chap1}

%BEGIN THE SECOND CHAPTER WHICH IS IN FILE DISS_chap2.tex IN THE FILE SPACE.
\input{chapter/DISS_chap2}


%BEGIN THE THIRD CHAPTER WHICH IS IN FILE DISS_chap3.tex IN THE FILE SPACE.
\input{chapter/DISS_chap3}

%BEGIN THE FOURTH CHAPTER. THIS SHOWS THE USE OF FONTS
%\input{chapter/DISS_chap4}

%BEGIN THE FIFTH CHAPTER. THIS IS TAKE FROM testmath.tex FROM THE AMS PACKAGE.
% TO ILLUSTRATE THE USE OF amslatex.

%\input{chapter/DISS_chap5}


%JK REV 7 MODIFIED ENTRY FOR TABLE OF CONENTS FOR APPENDIX.
\addtocontents{toc}{\addvspace{20pt}}
\addtocontents{toc}{\hspace*{-17pt}{\bf APPENDIX} \hfill }

\input{chapter/DISS_appenda}


%+++++++++++++++++++++++++PROCESS THE BIBLIOGRAPHY++++++++++++++++++++++++++++++





%PROCESS THE BIBLIOGRAPHY
% REVIEW THE DOCUMENTATION IN ./doc.
\baselineskip=12pt


%THE NOCITE COMMAND IS NEEDED TO INPUT REFERENCES WHICH ARE NOT EXPLICITELY
%REFERENCED IN THE BODY OF THE TEXT USING THE CITE COMMAND.



%REV 7
% don't use any of these... ... .. Kolibal refs


%THE NOCITE COMMAND IS NEEDED TO INPUT REFERENCES WHICH ARE NOT EXPLICITELY
%REFERENCED IN THE BODY OF THE TEXT USING THE CITE COMMAND.
%\nocite{mayt}
%\nocite{rozdestvenskiiandjanenko}
%\nocite{Bergermj}
%\nocite{gustafssonb}
%\nocite{abbettmj}
%\nocite{kreissho}
%\nocite{wickelgrenwa}
%\nocite{sodga}
%\nocite{laxpd}
%\nocite{wagons}
%\nocite{bowersandwilson}
%\nocite{krutzm}
%\nocite{lamarshjr}
%\nocite{walkerandmiller}
%\nocite{profioae}
%\nocite{grantpj}
%\nocite{mccauslandi}


{\small
%\baselineskip=12.8pt




%THIS IS OPTIONAL. IT WILL PUT A LISTING OF THE BIBLIOGRAPHY IN THE
% TABLE OF CONTENTS.
\clearpage
\addtocontents{toc}{ \addvspace{15pt}}
\addtocontents{toc}{\hspace*{-17pt}{\bf BIBLIOGRAPHY}
           \dtfil \hspace{10pt} \arabic{page}}
\addtocontents{toc}{ \addvspace{10pt}}


%YOU MAY USE plain OR siam STYLE
\bibliography{DISSc}
%\bibliographystyle{siam}
\bibliographystyle{plain}




%THIS IS OPTIONAL. IT WILL PUT A LISTING OF THE INDEX IN THE
% TABLE OF CONTENTS.
\clearpage
\addtocontents{toc}{ \addvspace{15pt}}
\addtocontents{toc}{\hspace*{-17pt}{\bf INDEX}
           \dtfil \hspace{10pt} \arabic{page}}
\addtocontents{toc}{ \addvspace{10pt}}







}


%++++++++++++++++++++++END PROCESS THE BIBLIOGRAPHY+++++++++++++++++++++++++++++



%++++++++++++++++++++++++++++PROCESS THE INDEX++++++++++++++++++++++++++++++++++
%PROCESS THE INDEX.
% REVIEW THE DOCUMENTATION IN ./doc.

%THIS CHOOSES A WIDE SINGLE SPACE FOR THE INDEX.
\baselineskip=15pt

%~~~~~~~~~~~~~~~~~~~~~~~~~~~~~~~~~~~~~~~~~~~~~~~~~~~~~~~~~~~~~~~~~~~~~~~~~~~~~~~
%DOCUMENT:      DISSc.tex
%AUTHOR:    J. Kolibal
%REV:           26
%DATE:          Mon Nov 21 09:40:52 CST 2005


%
%
%DEPENDENCIES: Style files and bibliography style files:
%               dissertation_usm.cls  DISS_pream.tex
%               plain.bst
%
%              Chapters in the dissertation:
%               DISS_chap1.tex DISS_chap2.tex DISS_chap3.tex
%                DISS_chap4.tex DISS_chap5.tex DISS_chap6.tex  DISS_output.tex
%
%              The bibliography database:
%               DISSc.bib
%
%              Disseration required lead pages:
%               DISS_abstract.tex DISS_abstract_titlepage.tex
%               DISS_titlepage.tex DISS_abbreviations.tex
%
%              Figures:
%               DISS_mach.PS DISS_conv.PS DISS_machgnu.PS
%
%              Verbatim computer input:
%               DISS_output.tex
%
%PURPOSE:      To demonstrate the generation of a dissertation for
%           for Scientific Computing and Mathematics at USM.
%
%APPROACH:     This uses modified class files based on book.cls files
%               to control the layout of the page for the majority of
%               the page appearance. Some commands are not hard coded
%               as macros in the style file. These must be entered
%               as shown in this example in order to produce a file which
%               is consistent with the requirments of the Graduate College.
%
%               You must also be aware of the rules that LaTeX uses to
%               set a page. This includes 1) each paragraph begins on a
%               newline after a blank line; 2) a word begins when at least
%               one blank space is left on a line (punctuation belongs
%               to the end of words, e.g.,  'this sentence ends. ' is
%               the way to type, not 'this sentence ends .'; 3) parentheses
%               belong to the words internal to the bracketed phrase,
%               e.g., 'this (while they said otherwise) and not those'
%               in contrast to 'this( while they said otherwise )and not those';
%               and, 4) remember to tie together objects with forced
%               spaces, thus 'Fig.~5' and not 'Fig. 5'. The detail
%               is important to getting TeX to interpret the spacing.
%
%
%+++++++++++++++++++++THIS VERSION IS SETUP TO RUN ON LINUX+++++++++++++++++++++

%
% You need to run teTeX (LaTeX2e) with the paths set correctly as
%TEXINPUTS=.:$HOME/common/defaults/latex/inputs//:/home/defaults/latex/inputs//:
%
% If you are at USM on any Scientific Computing workstation running  Linux,
% the setup of redhat linux should work without any additions modifations.
%
%++++++++++++++++++++++SETUP THE MARGINS AND SPACING++++++++++++++++++++++++++++

% THIS SETUP INCLUDES THE AMSLatex FONTS AND MACROS AND THE NEW graphicx PACKAGE
\documentclass[oneside,12pt]{dissertation_usm}
\usepackage{graphicx,amssymb,amsfonts,amsmath,amsthm,eucal}
% THE eucal package provides improved math caligraphic fonts.

% PROVIDES AN ALTERNATIVE FONT ENCODING.
%\usepackage[T1]{fontenc}

%THIS USES THE Times-Roman FONTS (DEFAULT IS TO USE THIS PACKAGE)
\usepackage{mathptm}

% SELECT GLOSSARY ---
%THIS PACKAGE ALLOWS YOU TO CREATE GLOSSARY ENTRIES. DO NOT UNCOMMENT EVEN
% IF YOU PLAN TO NOT USE.
\usepackage{nomencl}
% SELECT GLOSSARY ---


%THIS ALLOWS YOU TO CHANGE THE SCALE OF YOUR OUTPUT (THE UPPER LEFT CORNER
% IS HELD CONSTANT). DO NOT USE UNLESS YOU NEED TO SCALE YOUR OUTPUT.
% CHANGE THE \mag=VALUE command in the mag.sty FILE TO CHANGE THE SCALE.
%\mag=1000 IS NORMAL SCALING.
%\usepackage{mag}
%



%PROVIDES COLOR FOR TEXT AND BACKGROUND.
\usepackage{color}


%% THIS SETUP IS FOR Latex WITHOUT ANY AMSLatex.
%\documentclass[oneside,12pt]{dissertation_usm}

%\usepackage{graphicx}


%
\setlength{\topmargin}{-0.1in}
%\setlength{\textheight}{8.9in}
\setlength{\textheight}{9.3in}
\setlength{\textwidth}{5.9in}
\setlength{\oddsidemargin}{0.5in}
\setlength{\evensidemargin}{0.5in}
%






%YOU MUST BRING IN THE epsf.tex FILE IF YOU USE epsfbox TO INPUT YOUR FIGURES.
\input{epsf}


%\sloppy
%\raggedbottom



%DEFINE COMMANDS. THESE ARE GLOBAL. USE THE DISS_pream_ams IF YOU ARE USING
% AMS LaTeX COMMANDS.
\input{preamble/DISS_pream}
\input{preamble/DISS_pream_ams}



%THIS WILL ALLOW YOU TO MAKE AN INDEX. TAKE THIS COMMAND OUT IF YOU DO
% NOT DESIRE AN INDEX.
\makeindex


%SELECT GLOSSARY ---
%THIS COMMAND WILL ALLOW YOU TO MAKE A GLOSSARY.
% DO NOT USE UNLESS YOU HAVE READ THE DOCUMENTATION FOR THE nomenclature.sty
% PACKAGE.
\refpage
\makeglossary
%SELECT GLOSSARY ---







\begin{document}




%THE TEXT OF THE DOCUMENT.





%DEFINE THE LEADING PAGES.
%THIS IS THE ABSTRACT TITLE PAGE.
\thispagestyle{empty}
\input{preamble/DISS_abstract_titlepage}
\setlength{\textheight}{8.7in}
\newpage


%THIS IS THE ABSTRACT.
\setcounter{page}{1}   %THIS PAGE MUST HAVE AN ARABIC NUMBER 1 ON IT
\thispagestyle{plain}
\input{preamble/DISS_abstract}
\newpage


%THIS IS THE COPYRIGHT PAGE.
\setcounter{page}{1}
\thispagestyle{empty}
\input{preamble/DISS_copyright}
\newpage




%THIS IS THE TITLE PAGE.
\thispagestyle{empty}
\input{preamble/DISS_titlepage}
\newpage


% THE ROMAN NUMBER OF THE PAGES BEGINS HERE
% THIS IS THE SECOND PAGE (ii) OF THE DOCUMENT.
{\pagestyle{plain}
\pagenumbering{roman}
\setcounter{page}{2}

%DEFINE THE LEADING PAGES.
%THIS IS THE DEDICATION PAGE.
% THIS IS THE SECOND PAGE (ii) OF THE DOCUMENT.
\input{preamble/DISS_dedication}
\newpage

%THIS IS THE TABLE OF CONTENTS.
% THIS IS THE SECOND PAGE (ii) OF THE DOCUMENT IF THERE
% IS NO DEDICATION PAGE.
%\thispagestyle{plain}
%\pagenumbering{roman}
%\setcounter{page}{2}
\tableofcontents

%ADDED ABSTRACT PAGE------------------------------------------------------------
\newcounter{abspage}
\setcounter{abspage}{1}
\addtocontents{toc}{ \addvspace{15pt}}

%PULLED TOC ENTRIES FLUSH.
\addtocontents{toc}{ {\bf ABSTRACT}
      \dtfil \hspace*{10pt}  \arabic{abspage}}
\addtocontents{toc}
%ADDED ABSTRACT PAGE------------------------------------------------------------


%THIS IS THE LISTING OF THE ACKNOWLEDGEMENTS
\newcounter{ackpage}
\setcounter{ackpage}{2}
\addtocontents{toc}{ \addvspace{15pt}}

%PULLED TOC ENTRIES FLUSH.
\addtocontents{toc}{ {\bf \hspace*{-21pt} ACKNOWLEDGEMENTS}
      \dtfil   \roman{ackpage}}
\addtocontents{toc} { \addvspace{15pt}}


% * * * MODIFIED listoffigures AND listoftables * * * -------------------------
%PULLED TOC ENTRIES FLUSH.
%Patch if page number of illustration entry is incorrect
\addtocounter{page}{1}
\addtocontents{toc}{\hspace*{-21pt} {\bf LIST OF ILLUSTRATIONS}
      \dtfil \hspace*{10pt}  \roman{page}}
\addtocontents{toc}{ \addvspace{15pt}}
\addtocounter{page}{-1}
\listoffigures


% PULLED TOC ENTRIES FLUSH.
%Patch if page number of tables entry is incorrect
\addtocounter{page}{1}
\addtocontents{toc}{ \addvspace{15pt}}
\addtocontents{toc}{\hspace*{-17pt}{\bf LIST OF TABLES}
       \dtfil \hspace{10pt} \roman{page}}
\addtocounter{page}{-1}
\listoftables
% * * * MODIFIED listoffigures AND listoftables * * * -------------------------


%THIS IS THE LIST OF ABBREVIATIONS.
\clearpage
\thispagestyle{plain}

%PULLED TOC ENTRIES FLUSH.
\addtocontents{toc}{ \addvspace{15pt}}
\addtocontents{toc}{\hspace*{-17pt}{\bf LIST OF ABBREVIATIONS}
           \dtfil \hspace{10pt} \roman{page}}
\addtocontents{toc}{ \addvspace{10pt}}

\input{preamble/DISS_abbreviations}


%SELECT GLOSSARY ---
%THIS IS THE NOMENCLATURE GLOSSARY
\newcounter{glosscount}
%\pagestyle{myheadings}
%\markboth{\small \sl \hfill NOTATION \hfill }{\small \sl \hfill NOTATION  \hfill}
%\setcounter{glosscount}{\thepage}
%\setcounter{glosscount}{13}
\setcounter{chapter}{0}

\chapter*{NOTATION AND GLOSSARY}
\section*{General Usage and Terminology}
\thispagestyle{plain}
The notation used in this text represents fairly standard mathematical
and computational usage. In many cases these fields tend to use
different preferred notation to indicate the same concept, and these
have been reconciled to the extent possible,
given the interdisciplinary nature of the material.
In particular, the notation for partial
derivatives varies extensively, and the notation used is chosen
for stylistic convenience based on the application. While it would be
convenient to utilize a standard nomenclature for this important
symbol, the many alternatives currently in the published literature
will continue to be utilized.



The blackboard fonts are used to denote standard
sets of numbers: $\real$ for the field of real numbers,
$\complex$ for the complex field, $\integer$ for the integers,
and $\rational$ for the rationals.
The capital letters, $A, B, \cdots $ are used to denote matrices,
including capital greek letters, e.g., $\Lambda$ for a diagnonal matrix.
Functions which are denoted in boldface type typically represent 
vector valued functions, and real valued functions usually are set in
lower case roman or greek letters.
Caligraphic letters, e.g., $\cal V$, are used to denote spaces such
as $\cal V$ denoting a vector space, $\cal H$ denoting a Hilbert space,
or $\cal F$ denoting a general function space.
Lower case letters such as $i, j, k, l, m, n$ and sometimes $p$  and $d$
are used to denote indices.

Vectors are typset in square brackets, e.g., $[ \cdot ]$, and matrices are
typeset in parenthesese, e.g., $( \cdot)$. 
In general the norms are typeset using double pairs of
lines, e.g., $|| \cdot ||$,
and the abolute value of numbers is denoted using a single pairs of lines,
e.g., $| \cdot |$.
Single pairs of lines around matrices indicates the determinant of the
matrix.

%\newpage

\glossfont




% THE \label{glosspage} COMMAND MAY HAVE TO BE SET AFTER THE GLOSSARY IS CALLED.
\label{glosspage}
\printglossary[1.2cm]             % SET TO 1.2cm wide.
%\setcounter{glosscount}{13}
%SET PAGE MANUALLY
%ADD GLOSSARY TO TABLE OF CONTENTS
\addtocontents{toc}{ \addvspace{0pt}}
\addtocontents{toc}{\hspace*{-17pt}{\bf NOTATION AND GLOSSARY}
%      \dtfil \hspace*{10pt}  \roman{glosscount}}
      \dtfil \hspace*{10pt}  \pageref{glosspage}}
\addtocontents{toc}{ \addvspace{15pt}}
%SELECT GLOSSARY ---
%\label{glosspage}
%\clearpage
\thispagestyle{empty}



}                                           % CLOSE PLAIN PAGE STYLE.


%THIS ENDS THE PREAMBLE FILES.



%....................................................................................
%THIS CLEARPAGE IS NECESSARY TO CLEAR THE BUFFERS
\clearpage
\baselineskip=18pt


%BEGIN THE TEXT OF THE DOCUMENT.
\pagenumbering{arabic}
\setcounter{page}{1}


%BEGIN THE FIRST CHAPTER WHICH IS IN FILE DISS_chap1.tex IN THE FILE SPACE.
\input{chapter/DISS_chap1}

%BEGIN THE SECOND CHAPTER WHICH IS IN FILE DISS_chap2.tex IN THE FILE SPACE.
\input{chapter/DISS_chap2}


%BEGIN THE THIRD CHAPTER WHICH IS IN FILE DISS_chap3.tex IN THE FILE SPACE.
\input{chapter/DISS_chap3}

%BEGIN THE FOURTH CHAPTER. THIS SHOWS THE USE OF FONTS
%\input{chapter/DISS_chap4}

%BEGIN THE FIFTH CHAPTER. THIS IS TAKE FROM testmath.tex FROM THE AMS PACKAGE.
% TO ILLUSTRATE THE USE OF amslatex.

%\input{chapter/DISS_chap5}


%JK REV 7 MODIFIED ENTRY FOR TABLE OF CONENTS FOR APPENDIX.
\addtocontents{toc}{\addvspace{20pt}}
\addtocontents{toc}{\hspace*{-17pt}{\bf APPENDIX} \hfill }

\input{chapter/DISS_appenda}


%+++++++++++++++++++++++++PROCESS THE BIBLIOGRAPHY++++++++++++++++++++++++++++++





%PROCESS THE BIBLIOGRAPHY
% REVIEW THE DOCUMENTATION IN ./doc.
\baselineskip=12pt


%THE NOCITE COMMAND IS NEEDED TO INPUT REFERENCES WHICH ARE NOT EXPLICITELY
%REFERENCED IN THE BODY OF THE TEXT USING THE CITE COMMAND.



%REV 7
% don't use any of these... ... .. Kolibal refs


%THE NOCITE COMMAND IS NEEDED TO INPUT REFERENCES WHICH ARE NOT EXPLICITELY
%REFERENCED IN THE BODY OF THE TEXT USING THE CITE COMMAND.
%\nocite{mayt}
%\nocite{rozdestvenskiiandjanenko}
%\nocite{Bergermj}
%\nocite{gustafssonb}
%\nocite{abbettmj}
%\nocite{kreissho}
%\nocite{wickelgrenwa}
%\nocite{sodga}
%\nocite{laxpd}
%\nocite{wagons}
%\nocite{bowersandwilson}
%\nocite{krutzm}
%\nocite{lamarshjr}
%\nocite{walkerandmiller}
%\nocite{profioae}
%\nocite{grantpj}
%\nocite{mccauslandi}


{\small
%\baselineskip=12.8pt




%THIS IS OPTIONAL. IT WILL PUT A LISTING OF THE BIBLIOGRAPHY IN THE
% TABLE OF CONTENTS.
\clearpage
\addtocontents{toc}{ \addvspace{15pt}}
\addtocontents{toc}{\hspace*{-17pt}{\bf BIBLIOGRAPHY}
           \dtfil \hspace{10pt} \arabic{page}}
\addtocontents{toc}{ \addvspace{10pt}}


%YOU MAY USE plain OR siam STYLE
\bibliography{DISSc}
%\bibliographystyle{siam}
\bibliographystyle{plain}




%THIS IS OPTIONAL. IT WILL PUT A LISTING OF THE INDEX IN THE
% TABLE OF CONTENTS.
\clearpage
\addtocontents{toc}{ \addvspace{15pt}}
\addtocontents{toc}{\hspace*{-17pt}{\bf INDEX}
           \dtfil \hspace{10pt} \arabic{page}}
\addtocontents{toc}{ \addvspace{10pt}}







}


%++++++++++++++++++++++END PROCESS THE BIBLIOGRAPHY+++++++++++++++++++++++++++++



%++++++++++++++++++++++++++++PROCESS THE INDEX++++++++++++++++++++++++++++++++++
%PROCESS THE INDEX.
% REVIEW THE DOCUMENTATION IN ./doc.

%THIS CHOOSES A WIDE SINGLE SPACE FOR THE INDEX.
\baselineskip=15pt

%~~~~~~~~~~~~~~~~~~~~~~~~~~~~~~~~~~~~~~~~~~~~~~~~~~~~~~~~~~~~~~~~~~~~~~~~~~~~~~~
%DOCUMENT:      DISSc.tex
%AUTHOR:    J. Kolibal
%REV:           26
%DATE:          Mon Nov 21 09:40:52 CST 2005


%
%
%DEPENDENCIES: Style files and bibliography style files:
%               dissertation_usm.cls  DISS_pream.tex
%               plain.bst
%
%              Chapters in the dissertation:
%               DISS_chap1.tex DISS_chap2.tex DISS_chap3.tex
%                DISS_chap4.tex DISS_chap5.tex DISS_chap6.tex  DISS_output.tex
%
%              The bibliography database:
%               DISSc.bib
%
%              Disseration required lead pages:
%               DISS_abstract.tex DISS_abstract_titlepage.tex
%               DISS_titlepage.tex DISS_abbreviations.tex
%
%              Figures:
%               DISS_mach.PS DISS_conv.PS DISS_machgnu.PS
%
%              Verbatim computer input:
%               DISS_output.tex
%
%PURPOSE:      To demonstrate the generation of a dissertation for
%           for Scientific Computing and Mathematics at USM.
%
%APPROACH:     This uses modified class files based on book.cls files
%               to control the layout of the page for the majority of
%               the page appearance. Some commands are not hard coded
%               as macros in the style file. These must be entered
%               as shown in this example in order to produce a file which
%               is consistent with the requirments of the Graduate College.
%
%               You must also be aware of the rules that LaTeX uses to
%               set a page. This includes 1) each paragraph begins on a
%               newline after a blank line; 2) a word begins when at least
%               one blank space is left on a line (punctuation belongs
%               to the end of words, e.g.,  'this sentence ends. ' is
%               the way to type, not 'this sentence ends .'; 3) parentheses
%               belong to the words internal to the bracketed phrase,
%               e.g., 'this (while they said otherwise) and not those'
%               in contrast to 'this( while they said otherwise )and not those';
%               and, 4) remember to tie together objects with forced
%               spaces, thus 'Fig.~5' and not 'Fig. 5'. The detail
%               is important to getting TeX to interpret the spacing.
%
%
%+++++++++++++++++++++THIS VERSION IS SETUP TO RUN ON LINUX+++++++++++++++++++++

%
% You need to run teTeX (LaTeX2e) with the paths set correctly as
%TEXINPUTS=.:$HOME/common/defaults/latex/inputs//:/home/defaults/latex/inputs//:
%
% If you are at USM on any Scientific Computing workstation running  Linux,
% the setup of redhat linux should work without any additions modifations.
%
%++++++++++++++++++++++SETUP THE MARGINS AND SPACING++++++++++++++++++++++++++++

% THIS SETUP INCLUDES THE AMSLatex FONTS AND MACROS AND THE NEW graphicx PACKAGE
\documentclass[oneside,12pt]{dissertation_usm}
\usepackage{graphicx,amssymb,amsfonts,amsmath,amsthm,eucal}
% THE eucal package provides improved math caligraphic fonts.

% PROVIDES AN ALTERNATIVE FONT ENCODING.
%\usepackage[T1]{fontenc}

%THIS USES THE Times-Roman FONTS (DEFAULT IS TO USE THIS PACKAGE)
\usepackage{mathptm}

% SELECT GLOSSARY ---
%THIS PACKAGE ALLOWS YOU TO CREATE GLOSSARY ENTRIES. DO NOT UNCOMMENT EVEN
% IF YOU PLAN TO NOT USE.
\usepackage{nomencl}
% SELECT GLOSSARY ---


%THIS ALLOWS YOU TO CHANGE THE SCALE OF YOUR OUTPUT (THE UPPER LEFT CORNER
% IS HELD CONSTANT). DO NOT USE UNLESS YOU NEED TO SCALE YOUR OUTPUT.
% CHANGE THE \mag=VALUE command in the mag.sty FILE TO CHANGE THE SCALE.
%\mag=1000 IS NORMAL SCALING.
%\usepackage{mag}
%



%PROVIDES COLOR FOR TEXT AND BACKGROUND.
\usepackage{color}


%% THIS SETUP IS FOR Latex WITHOUT ANY AMSLatex.
%\documentclass[oneside,12pt]{dissertation_usm}

%\usepackage{graphicx}


%
\setlength{\topmargin}{-0.1in}
%\setlength{\textheight}{8.9in}
\setlength{\textheight}{9.3in}
\setlength{\textwidth}{5.9in}
\setlength{\oddsidemargin}{0.5in}
\setlength{\evensidemargin}{0.5in}
%






%YOU MUST BRING IN THE epsf.tex FILE IF YOU USE epsfbox TO INPUT YOUR FIGURES.
\input{epsf}


%\sloppy
%\raggedbottom



%DEFINE COMMANDS. THESE ARE GLOBAL. USE THE DISS_pream_ams IF YOU ARE USING
% AMS LaTeX COMMANDS.
\input{preamble/DISS_pream}
\input{preamble/DISS_pream_ams}



%THIS WILL ALLOW YOU TO MAKE AN INDEX. TAKE THIS COMMAND OUT IF YOU DO
% NOT DESIRE AN INDEX.
\makeindex


%SELECT GLOSSARY ---
%THIS COMMAND WILL ALLOW YOU TO MAKE A GLOSSARY.
% DO NOT USE UNLESS YOU HAVE READ THE DOCUMENTATION FOR THE nomenclature.sty
% PACKAGE.
\refpage
\makeglossary
%SELECT GLOSSARY ---







\begin{document}




%THE TEXT OF THE DOCUMENT.





%DEFINE THE LEADING PAGES.
%THIS IS THE ABSTRACT TITLE PAGE.
\thispagestyle{empty}
\input{preamble/DISS_abstract_titlepage}
\setlength{\textheight}{8.7in}
\newpage


%THIS IS THE ABSTRACT.
\setcounter{page}{1}   %THIS PAGE MUST HAVE AN ARABIC NUMBER 1 ON IT
\thispagestyle{plain}
\input{preamble/DISS_abstract}
\newpage


%THIS IS THE COPYRIGHT PAGE.
\setcounter{page}{1}
\thispagestyle{empty}
\input{preamble/DISS_copyright}
\newpage




%THIS IS THE TITLE PAGE.
\thispagestyle{empty}
\input{preamble/DISS_titlepage}
\newpage


% THE ROMAN NUMBER OF THE PAGES BEGINS HERE
% THIS IS THE SECOND PAGE (ii) OF THE DOCUMENT.
{\pagestyle{plain}
\pagenumbering{roman}
\setcounter{page}{2}

%DEFINE THE LEADING PAGES.
%THIS IS THE DEDICATION PAGE.
% THIS IS THE SECOND PAGE (ii) OF THE DOCUMENT.
\input{preamble/DISS_dedication}
\newpage

%THIS IS THE TABLE OF CONTENTS.
% THIS IS THE SECOND PAGE (ii) OF THE DOCUMENT IF THERE
% IS NO DEDICATION PAGE.
%\thispagestyle{plain}
%\pagenumbering{roman}
%\setcounter{page}{2}
\tableofcontents

%ADDED ABSTRACT PAGE------------------------------------------------------------
\newcounter{abspage}
\setcounter{abspage}{1}
\addtocontents{toc}{ \addvspace{15pt}}

%PULLED TOC ENTRIES FLUSH.
\addtocontents{toc}{ {\bf ABSTRACT}
      \dtfil \hspace*{10pt}  \arabic{abspage}}
\addtocontents{toc}
%ADDED ABSTRACT PAGE------------------------------------------------------------


%THIS IS THE LISTING OF THE ACKNOWLEDGEMENTS
\newcounter{ackpage}
\setcounter{ackpage}{2}
\addtocontents{toc}{ \addvspace{15pt}}

%PULLED TOC ENTRIES FLUSH.
\addtocontents{toc}{ {\bf \hspace*{-21pt} ACKNOWLEDGEMENTS}
      \dtfil   \roman{ackpage}}
\addtocontents{toc} { \addvspace{15pt}}


% * * * MODIFIED listoffigures AND listoftables * * * -------------------------
%PULLED TOC ENTRIES FLUSH.
%Patch if page number of illustration entry is incorrect
\addtocounter{page}{1}
\addtocontents{toc}{\hspace*{-21pt} {\bf LIST OF ILLUSTRATIONS}
      \dtfil \hspace*{10pt}  \roman{page}}
\addtocontents{toc}{ \addvspace{15pt}}
\addtocounter{page}{-1}
\listoffigures


% PULLED TOC ENTRIES FLUSH.
%Patch if page number of tables entry is incorrect
\addtocounter{page}{1}
\addtocontents{toc}{ \addvspace{15pt}}
\addtocontents{toc}{\hspace*{-17pt}{\bf LIST OF TABLES}
       \dtfil \hspace{10pt} \roman{page}}
\addtocounter{page}{-1}
\listoftables
% * * * MODIFIED listoffigures AND listoftables * * * -------------------------


%THIS IS THE LIST OF ABBREVIATIONS.
\clearpage
\thispagestyle{plain}

%PULLED TOC ENTRIES FLUSH.
\addtocontents{toc}{ \addvspace{15pt}}
\addtocontents{toc}{\hspace*{-17pt}{\bf LIST OF ABBREVIATIONS}
           \dtfil \hspace{10pt} \roman{page}}
\addtocontents{toc}{ \addvspace{10pt}}

\input{preamble/DISS_abbreviations}


%SELECT GLOSSARY ---
%THIS IS THE NOMENCLATURE GLOSSARY
\newcounter{glosscount}
%\pagestyle{myheadings}
%\markboth{\small \sl \hfill NOTATION \hfill }{\small \sl \hfill NOTATION  \hfill}
\input{glossary}
% THE \label{glosspage} COMMAND MAY HAVE TO BE SET AFTER THE GLOSSARY IS CALLED.
\label{glosspage}
\printglossary[1.2cm]             % SET TO 1.2cm wide.
%\setcounter{glosscount}{13}
%SET PAGE MANUALLY
%ADD GLOSSARY TO TABLE OF CONTENTS
\addtocontents{toc}{ \addvspace{0pt}}
\addtocontents{toc}{\hspace*{-17pt}{\bf NOTATION AND GLOSSARY}
%      \dtfil \hspace*{10pt}  \roman{glosscount}}
      \dtfil \hspace*{10pt}  \pageref{glosspage}}
\addtocontents{toc}{ \addvspace{15pt}}
%SELECT GLOSSARY ---
%\label{glosspage}
%\clearpage
\thispagestyle{empty}



}                                           % CLOSE PLAIN PAGE STYLE.


%THIS ENDS THE PREAMBLE FILES.



%....................................................................................
%THIS CLEARPAGE IS NECESSARY TO CLEAR THE BUFFERS
\clearpage
\baselineskip=18pt


%BEGIN THE TEXT OF THE DOCUMENT.
\pagenumbering{arabic}
\setcounter{page}{1}


%BEGIN THE FIRST CHAPTER WHICH IS IN FILE DISS_chap1.tex IN THE FILE SPACE.
\input{chapter/DISS_chap1}

%BEGIN THE SECOND CHAPTER WHICH IS IN FILE DISS_chap2.tex IN THE FILE SPACE.
\input{chapter/DISS_chap2}


%BEGIN THE THIRD CHAPTER WHICH IS IN FILE DISS_chap3.tex IN THE FILE SPACE.
\input{chapter/DISS_chap3}

%BEGIN THE FOURTH CHAPTER. THIS SHOWS THE USE OF FONTS
%\input{chapter/DISS_chap4}

%BEGIN THE FIFTH CHAPTER. THIS IS TAKE FROM testmath.tex FROM THE AMS PACKAGE.
% TO ILLUSTRATE THE USE OF amslatex.

%\input{chapter/DISS_chap5}


%JK REV 7 MODIFIED ENTRY FOR TABLE OF CONENTS FOR APPENDIX.
\addtocontents{toc}{\addvspace{20pt}}
\addtocontents{toc}{\hspace*{-17pt}{\bf APPENDIX} \hfill }

\input{chapter/DISS_appenda}


%+++++++++++++++++++++++++PROCESS THE BIBLIOGRAPHY++++++++++++++++++++++++++++++





%PROCESS THE BIBLIOGRAPHY
% REVIEW THE DOCUMENTATION IN ./doc.
\baselineskip=12pt


%THE NOCITE COMMAND IS NEEDED TO INPUT REFERENCES WHICH ARE NOT EXPLICITELY
%REFERENCED IN THE BODY OF THE TEXT USING THE CITE COMMAND.
\input{DISSc_nocite}

{\small
%\baselineskip=12.8pt




%THIS IS OPTIONAL. IT WILL PUT A LISTING OF THE BIBLIOGRAPHY IN THE
% TABLE OF CONTENTS.
\clearpage
\addtocontents{toc}{ \addvspace{15pt}}
\addtocontents{toc}{\hspace*{-17pt}{\bf BIBLIOGRAPHY}
           \dtfil \hspace{10pt} \arabic{page}}
\addtocontents{toc}{ \addvspace{10pt}}


%YOU MAY USE plain OR siam STYLE
\bibliography{DISSc}
%\bibliographystyle{siam}
\bibliographystyle{plain}




%THIS IS OPTIONAL. IT WILL PUT A LISTING OF THE INDEX IN THE
% TABLE OF CONTENTS.
\clearpage
\addtocontents{toc}{ \addvspace{15pt}}
\addtocontents{toc}{\hspace*{-17pt}{\bf INDEX}
           \dtfil \hspace{10pt} \arabic{page}}
\addtocontents{toc}{ \addvspace{10pt}}







}


%++++++++++++++++++++++END PROCESS THE BIBLIOGRAPHY+++++++++++++++++++++++++++++



%++++++++++++++++++++++++++++PROCESS THE INDEX++++++++++++++++++++++++++++++++++
%PROCESS THE INDEX.
% REVIEW THE DOCUMENTATION IN ./doc.

%THIS CHOOSES A WIDE SINGLE SPACE FOR THE INDEX.
\baselineskip=15pt

\input{DISSc.ind}


%+++++++++++++++++++++++++END PROCESS THE INDEX+++++++++++++++++++++++++++++++++




\end{document}



%+++++++++++++++++++++++++END PROCESS THE INDEX+++++++++++++++++++++++++++++++++




\end{document}



%+++++++++++++++++++++++++END PROCESS THE INDEX+++++++++++++++++++++++++++++++++




\end{document}



%+++++++++++++++++++++++++END PROCESS THE INDEX+++++++++++++++++++++++++++++++++




\end{document}
